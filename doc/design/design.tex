\pdfoutput=1
\synctex=1
\documentclass[11pt,a4paper,twoside]{article}

\usepackage{amsfonts,amsmath,amssymb}
\usepackage[vmargin=25mm,inner=15mm,outer=35mm]{geometry}
\usepackage{graphicx}
\usepackage{hypernat}
\usepackage[utf8]{inputenc}
\usepackage{microtype}
\usepackage{pdfsync}
\usepackage[hyperfootnotes=false]{hyperref}
\usepackage{textcomp}
\hypersetup{colorlinks,citecolor=black,urlcolor=blue}

\newcommand{\degree}{\ensuremath{^\circ}}%
\newcommand{\etal}{et al.}%
\newcommand{\ie}{i.\,e.}%


\markboth{\emph{Jochen K\"{u}pper: }CASS -- internal report \#200901}{Do not distribute}

\begin{document}


\title{CFEL-ASG Software System (CASS) \\
   Design notes for a general setup}
\author{Jochen Küpper}
\maketitle

\section{Introduction}
\label{sec:introduction}

The Max-Planck Advanced Study Group at the Center for Free Electron Laser Science has developed the
\emph{CFEL-ASG Multi Purpose} experimental chamber for use at upcoming X-ray Free-Electron-Laser
facilities. The setup includes several data-acquisition devices, including pnCCDs, VMI
spectrometers, and a COLTRIMS reaction microscope (REMI). It will also include a TOF for detecting
particle hits and a MCP for detecting the direct beam behind the second detector, thereby taking
into account the influence of the apertures.

In oder to provide the necessary online viewing capabilities for the upcoming experiments, a common
software system to obtain, massage, and analyze the data in online-mode is necessary. Here, we
present a description of a possible software layout.


\section{Requirements}
\label{sec:requirements}

The software system should be able to handle all relevant experimental data. This includes the
following:
\begin{itemize}
\item an arbitrary (currently 2) number of pnCCD detectors
\item an arbitrary (currently 2) numbers of Pulnix CCD (as used in the VMI)
\item Acqiris digitizers (as used in the REMI)
\item 2 XFEL intensity sensors (LCLS + CAMP)
\item YAG laser power meter
\end{itemize}
All data should be connected to a unique (and correct) timestamp; hopefully, LCLS will provide that,
otherwise we need to provide one ourselves.


\section{General layout of the software system}
\label{sec:general-layout}

At the most abstract level, a number of different tasks need to be performed, essentially
sequentially, with the data. The major tasks are
\begin{itemize}
\item merging and storing event data
\item data correction operations
   \begin{itemize}
   \item image manipulations (bad pixels, data reduction, \dots )
   \item Acqiris trace evaluation event coordinates
   \end{itemize}
\item serving events (and groups thereof)
\item data reduction
\item data picking
\item data averaging
\end{itemize}
We propose to split the individual tasks into separate libraries that, nevertheless, will run in the
same process for fast interoperability. The following individual libraries should be provided:

\begin{enumerate}
   \item event acquisition, merging, and storage
   \item pnCCD image manipulation
      \begin{itemize}\item 
      \item image corrections
      \item pixel binning
      \end{itemize}
   \item VMI image manipulation
   \item REMI data evaluation
\end{enumerate}
The tasks of data picking and data averaging shall be done in the main program. Others would,
optimally, also run as separate (sets of) threads in the main program to reach maximum performance,
but, in principle, coud also be separate processes.

\subsection{Special ``online-analysis''}
\label{sec:special-online-analysis}

\subsubsection{COLTRIMS reaction microscope (REMI)}
\label{sec:remi}

Parallel to the actions performed by the software system described here, Lutz Foucar will provide
his ``standard'' ROOT-based interpreted-C interface for online-analysis of REMI data. This will be a
standalone system, that gets the REMI data independently from LCLS and works solely by itself,
without any communication or synchronization with the GUI described in this document.

This package allows the user to easily add his own analysis code and let it run either interpreted
or compiled. Compilation is done via ``ACLIC''. In the code the user defines Histograms and fills
them according to his experimental interests. The user has then the ability to interact with these
histograms, such as making fits, find the Integral, etc via C-style ROOT-Macros.


\section{Implementation details}
\label{sec:implementation}

All code should be written in C or derivatives thereof (i.e., C++, Objective C), or languages that
have standardized and robust interfaces to C (i.e., Python). Individual parts should be written in a
single programming language, but different parts don't necessarily have to do so, if there is a C or
C++ interface for communication. Generally, all non-GUI libraries should be written in a platform
independent way, i.e., using standard language features unless
really impossible.

The central graphical user interface (GUI) will utilize the Qt library (version 4), mainly for its
platform independence. Moreover, LCLS also uses it and some synergies might arise.

We agreed that for non-GUI parts it is also ok to use Qt (QtCore), i.e., in the pnCCD library.
However, C++ standard library constructs are strongly preferred over Qt datatypes, but most
importantly is should be consistent and documented for individual parts (libraries) of the project.

Generally, hardware optimized code should be used. For example, for linear algebra optimized
BLAS/LAPACK shall be used (i.e., ATLAS, Apple's veclib), for image processing Apple's vImage vector
codes should work very well, \dots

\subsection{Communication}
\label{sec:communication}

Intra-process communication (using variables in a common address space) is much faster than any kind
of inter-process communication. As data transfer is foreseen to be the largest hurdle to overcome,
this really is the major point.

All individual parts of the CASS should be written as re-entrant/threadsafe libraries that are
linked into a single executable. The GUI will create a number of threads as necessary and forward
data between them. Data should not be transferred, but locked and only the respective pointers
transferred.

\subsection{Events}
\label{sec:events}

An event contains the following information for a single LCLS shot:

\begin{itemize}
\item bunchid, timestamp
\item all image data, all digitizer data
\item all diagnostics data:
   \begin{itemize}
   \item power meters (LCLS, YAG, \dots )
   \item beam position (from FIB patterns)
   \end{itemize}
\item all device settings
\end{itemize}


\section{Software sections and tasks}
\label{sec:software-sections-tasks}

In the following, the tasks performed by the individual parts are described in some more detail.


\subsection{Event server}
\label{sec:event-server}

The event server, which will run in an individual set of threads (or possibly, its own process) will
perform the following tasks sequentially on each individual event:

\begin{itemize}
\item get next events from the acquisition system (this could be ``LCLS'' temporary storage) and
   place it in event queue
\item determine Experiment-type (analog to run-specification) and store with event (hopefully, this
   will be done by LCLS)
\item call the data-picking/reduction library
\item if the event is still valid (nor ``reduced away'') mangle data (in parallel) according to
   parameters (from initialization)
   \begin{itemize}
   \item if event contains pnCCD images, call the pnCCD library to manipulate them
   \item if event contains VMI images, call the VMI library to manipulate them
   \item if event contains REMI data, call the REMI library to manipulate it (calculate images?)
   \end{itemize}
\item push event to connected clients
\end{itemize}


\subsubsection{Data picking and reduction}
\label{sec:data-pick-reduct}

Determine whether all simple measures match their requirements
\begin{itemize}
\item LCLS power meter value, YAG power meter value, \dots
\item remove unnecessary images: i.e., take away second pnCCD image if that's user-specified
\item potentially bin data (images pixel-wise, digitizer traces time-wise, \dots )
\end{itemize}

Merge data if possible, i.e., 

\begin{itemize}
\item merge images from front and back CCD into a single one (taking into account the corrections
   provided by the FIB patterns)
\end{itemize}


\subsubsection{Device libraries}
\label{sec:device-libraries}

\paragraph{pnCCD library}
\label{sec:pnccd-library}

programmatic Xonline/Racoon equivalent, this should provide
\begin{itemize}
\item parameter class
\item initialization routine, creating a state object from a parameter object
\item handle image according to state object (replace original object unless data-reduction binning
   is performed)
\end{itemize}
Such an interface could of course nicely be provided by a single C++ class for the work and a C++
class for the parameters.


\paragraph{VMI library}
\label{sec:vmi-library}

Is there much mangling to be done?
\begin{itemize}
\item parameter class
\item initialization routine, creating a state object from a parameter object
\item handle image according to state object (replace original object unless data-reduction binning
   is performed)
\end{itemize}


\paragraph{REMI library}
\label{sec:remi-library}

\begin{itemize}
\item parameter class
\item initialization routine, creating a state object from a parameter object
   \begin{itemize}
   \item this includes parameters from the GUI and
   \item parameters from LCLS ConfigV1 data passed along with the data stream
   \end{itemize}
\item handle data according to state object
   \begin{itemize}
   \item transform digitizer traces (waveforms) into 3D information for every event; i.e., create a
      vector of (x, y, t) tuples for every event (LCLS shot).
   \end{itemize}
\end{itemize}


\subsubsection{Averaging}
\label{sec:averaging}

\begin{itemize}
\item  Place current event in active set
\item  update averaged data according to run-definition
\item  remove inactive events
\end{itemize}

\subsection{Front-ends}
\label{sec:front-ends}

While all data-handling functionality will be pure non-graphical code, we will then provide a simple
yet efficient front-end as a modern user-interface. In principle, since these interfaces are so
slim, we could even provide several interfaces, adopted o various experiments. I even would suggest
multiple ``interfaces''

\subsubsection{Racoon}
\label{sec:racoon}

Provide the functionality necessary for commissioning the pnCCDs and to provide a parameter set for
the initialization of the pnCCD library.

This would most likely use Qt for the UI (because Racoon already does so)

\subsubsection{Diode}
\label{sec:diode}

General user interface for imaging experiments.

I would be happy to coordinate development of a common GUI for all experiments planned for 2009
using the Qt framework. This has the advantage that is platform independent and could also probably
re-use parts from existing tools.

The REMI data will be presented in a single 3D OpenGL view.


\subsubsection{REMI}
\label{sec:front-end-remi}

The ``standard'' ROOT based interpreted-C interface will be available.


\subsubsection{Others}
\label{sec:others}

Potentially, we might also want to have simple commissioning interfaces that work only for the VMI,
respectively. This should be based on the current GUIs as much as possible.

\clearpage

\section{Open questions and things to do}
\label{sec:open-quest-things}

\subsection{Data reduction}
\label{sec:data-reduction}

For which stages of the data-processing would it be useful to include efficient data-compression?
Obviously, going into the online-viewing chain, this format must also be highly efficient for
arithmetic and standard image processing.
\begin{itemize}
\item check with crystallographers (i.e., Wolfgang Kabsch)
\end{itemize}


\subsection{Simulation server}
\label{sec:simulation-server}

We need to write a server that simulates the data stream we will get from LCLS in order to test the
program in detail, including performance tests. This will still need to be specified, but, anyhow,
it must deliver data in exactly the same way as LCLS does. It should also run on a different
computer, to have real network communication between the server and CASS.


\begin{flushright}
\end{flushright}


\end{document}



%% Local Variables:
%% coding: utf-8
%% mode: flyspell
%% fill-column: 100
%% ispell-dictionary: "american"
%% reftex-cite-format: default
%% TeX-auto-save: t
%% TeX-close-quote: "''"
%% TeX-open-quote: "``"
%% TeX-parse-self: t
%% End:
